% Options for packages loaded elsewhere
\PassOptionsToPackage{unicode}{hyperref}
\PassOptionsToPackage{hyphens}{url}
\PassOptionsToPackage{dvipsnames,svgnames,x11names}{xcolor}
%
\documentclass[
  letterpaper,
  DIV=11,
  numbers=noendperiod]{scrartcl}

\usepackage{amsmath,amssymb}
\usepackage{iftex}
\ifPDFTeX
  \usepackage[T1]{fontenc}
  \usepackage[utf8]{inputenc}
  \usepackage{textcomp} % provide euro and other symbols
\else % if luatex or xetex
  \usepackage{unicode-math}
  \defaultfontfeatures{Scale=MatchLowercase}
  \defaultfontfeatures[\rmfamily]{Ligatures=TeX,Scale=1}
\fi
\usepackage{lmodern}
\ifPDFTeX\else  
    % xetex/luatex font selection
\fi
% Use upquote if available, for straight quotes in verbatim environments
\IfFileExists{upquote.sty}{\usepackage{upquote}}{}
\IfFileExists{microtype.sty}{% use microtype if available
  \usepackage[]{microtype}
  \UseMicrotypeSet[protrusion]{basicmath} % disable protrusion for tt fonts
}{}
\makeatletter
\@ifundefined{KOMAClassName}{% if non-KOMA class
  \IfFileExists{parskip.sty}{%
    \usepackage{parskip}
  }{% else
    \setlength{\parindent}{0pt}
    \setlength{\parskip}{6pt plus 2pt minus 1pt}}
}{% if KOMA class
  \KOMAoptions{parskip=half}}
\makeatother
\usepackage{xcolor}
\setlength{\emergencystretch}{3em} % prevent overfull lines
\setcounter{secnumdepth}{-\maxdimen} % remove section numbering
% Make \paragraph and \subparagraph free-standing
\makeatletter
\ifx\paragraph\undefined\else
  \let\oldparagraph\paragraph
  \renewcommand{\paragraph}{
    \@ifstar
      \xxxParagraphStar
      \xxxParagraphNoStar
  }
  \newcommand{\xxxParagraphStar}[1]{\oldparagraph*{#1}\mbox{}}
  \newcommand{\xxxParagraphNoStar}[1]{\oldparagraph{#1}\mbox{}}
\fi
\ifx\subparagraph\undefined\else
  \let\oldsubparagraph\subparagraph
  \renewcommand{\subparagraph}{
    \@ifstar
      \xxxSubParagraphStar
      \xxxSubParagraphNoStar
  }
  \newcommand{\xxxSubParagraphStar}[1]{\oldsubparagraph*{#1}\mbox{}}
  \newcommand{\xxxSubParagraphNoStar}[1]{\oldsubparagraph{#1}\mbox{}}
\fi
\makeatother


\providecommand{\tightlist}{%
  \setlength{\itemsep}{0pt}\setlength{\parskip}{0pt}}\usepackage{longtable,booktabs,array}
\usepackage{calc} % for calculating minipage widths
% Correct order of tables after \paragraph or \subparagraph
\usepackage{etoolbox}
\makeatletter
\patchcmd\longtable{\par}{\if@noskipsec\mbox{}\fi\par}{}{}
\makeatother
% Allow footnotes in longtable head/foot
\IfFileExists{footnotehyper.sty}{\usepackage{footnotehyper}}{\usepackage{footnote}}
\makesavenoteenv{longtable}
\usepackage{graphicx}
\makeatletter
\def\maxwidth{\ifdim\Gin@nat@width>\linewidth\linewidth\else\Gin@nat@width\fi}
\def\maxheight{\ifdim\Gin@nat@height>\textheight\textheight\else\Gin@nat@height\fi}
\makeatother
% Scale images if necessary, so that they will not overflow the page
% margins by default, and it is still possible to overwrite the defaults
% using explicit options in \includegraphics[width, height, ...]{}
\setkeys{Gin}{width=\maxwidth,height=\maxheight,keepaspectratio}
% Set default figure placement to htbp
\makeatletter
\def\fps@figure{htbp}
\makeatother

\KOMAoption{captions}{tableheading}
\makeatletter
\@ifpackageloaded{caption}{}{\usepackage{caption}}
\AtBeginDocument{%
\ifdefined\contentsname
  \renewcommand*\contentsname{Table of contents}
\else
  \newcommand\contentsname{Table of contents}
\fi
\ifdefined\listfigurename
  \renewcommand*\listfigurename{List of Figures}
\else
  \newcommand\listfigurename{List of Figures}
\fi
\ifdefined\listtablename
  \renewcommand*\listtablename{List of Tables}
\else
  \newcommand\listtablename{List of Tables}
\fi
\ifdefined\figurename
  \renewcommand*\figurename{Figure}
\else
  \newcommand\figurename{Figure}
\fi
\ifdefined\tablename
  \renewcommand*\tablename{Table}
\else
  \newcommand\tablename{Table}
\fi
}
\@ifpackageloaded{float}{}{\usepackage{float}}
\floatstyle{ruled}
\@ifundefined{c@chapter}{\newfloat{codelisting}{h}{lop}}{\newfloat{codelisting}{h}{lop}[chapter]}
\floatname{codelisting}{Listing}
\newcommand*\listoflistings{\listof{codelisting}{List of Listings}}
\makeatother
\makeatletter
\makeatother
\makeatletter
\@ifpackageloaded{caption}{}{\usepackage{caption}}
\@ifpackageloaded{subcaption}{}{\usepackage{subcaption}}
\makeatother

\ifLuaTeX
  \usepackage{selnolig}  % disable illegal ligatures
\fi
\usepackage{bookmark}

\IfFileExists{xurl.sty}{\usepackage{xurl}}{} % add URL line breaks if available
\urlstyle{same} % disable monospaced font for URLs
\hypersetup{
  pdftitle={Tutorial-08},
  pdfauthor={Kate Saunders},
  colorlinks=true,
  linkcolor={blue},
  filecolor={Maroon},
  citecolor={Blue},
  urlcolor={Blue},
  pdfcreator={LaTeX via pandoc}}


\title{Tutorial-08}
\author{Kate Saunders}
\date{}

\begin{document}
\maketitle

\renewcommand*\contentsname{Table of contents}
{
\hypersetup{linkcolor=}
\setcounter{tocdepth}{3}
\tableofcontents
}

\subsection{Visual Communication}\label{visual-communication}

\subsubsection{Learning Objectives}\label{learning-objectives}

Today we will look at examples of visualisation. We will practice
identifying if key messages:

\begin{itemize}
\item
  are appropriate for the audience
\item
  are they communicated well
\end{itemize}

To achieve this we will look at different examples of visual
communication.

\subsubsection{Preparation}\label{preparation}

\begin{itemize}
\tightlist
\item
  Take some time to read this
  \href{https://www.washingtonpost.com/immigration/2024/02/11/trump-biden-immigration-border-compared/}{article}
  by the Washington Post this article. If you can not access this
  article, do not pay for it. I haved include the important plots below,
  you can read the article text by accessing the article from the
  \href{https://www.proquest.com/docview/2924818088/D66F0B6AE5C14557PQ/1?accountid=12528&sourcetype=Blogs,\%20Podcasts,\%20&\%20Websites}{library
  here}.
\end{itemize}

\subsubsection{Motivation}\label{motivation}

This lecture we want to look at visual communication and discuss key
messages.

Recall this plot from Lecture 1, which has been used by President Trump
to show how the number of illegal immigration over time and with
different administrations.

\begin{center}
\includegraphics[width=4.16667in,height=\textheight]{../images/lecture-01/ImmigrationChart.png}
\end{center}

The Washington Post used 12 plots to explore this same issue. Using
these plots to provide more context to the rates of illegal immigration.

\paragraph{Some Context}\label{some-context}

\begin{itemize}
\item
  Immigration was a key issue in the 2024 presidential race.
\item
  Trump lost the 2020 presidential election to Biden in November, 2020.
\item
  Biden took the presidential office in January 2021.
\item
  Trump won the 2024 presidential election in November, 2024.
\end{itemize}

\subsubsection{Exercise 1}\label{exercise-1}

Discuss in your groups, the immigration plot used at Trump Republican
rally's:

\begin{enumerate}
\def\labelenumi{(\roman{enumi})}
\item
  Who is the audience for this plot?
\item
  What is the \textbf{key} message of this plot?
\item
  What are the important secondary messages?
\item
  Are the key messages clear and easy to identify?
\item
  How does the audience influence the perception of the key messages?
\end{enumerate}

\subsubsection{Exercise 2}\label{exercise-2}

Discuss in your group's the immigration plots shown by the Washington
Post.

\begin{enumerate}
\def\labelenumi{\roman{enumi}.}
\item
  Who is the audience for these plots?
\item
  What is the \emph{key} message for each of the below plots?
\end{enumerate}

\begin{itemize}
\tightlist
\item
  This plot shows the number of Illegal crossing at the US-Mexico border
  (2017 - 2023).
\end{itemize}

\includegraphics[width=4.16667in,height=\textheight]{images/illegal_border_crossings.png}

\begin{itemize}
\tightlist
\item
  This plot shows the number of single adults, families and
  unaccompanied minors.
\end{itemize}

\includegraphics[width=4.16667in,height=\textheight]{images/who_is_crossing.png}

\begin{itemize}
\tightlist
\item
  This plot shows the main countries of where illegal immigrants are
  coming from.
\end{itemize}

\includegraphics[width=4.16667in,height=\textheight]{images/where_people_came_from.png}

\begin{itemize}
\tightlist
\item
  This plot shows the number of people deported each year.
\end{itemize}

\includegraphics[width=4.16667in,height=\textheight]{images/deported.png}

\subsubsection{Note}\label{note}

In Assignment 2, you will combine your own set of plots to tell a data
story and share key messages.

\subparagraph{Material developed by Dr.~Kate
Saunders.}\label{material-developed-by-dr.-kate-saunders.}




\end{document}
